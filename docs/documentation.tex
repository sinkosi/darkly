\input{0.Extras/settings.tex}
\makeindex
\begin{document}

%----------------------------------------------------------------------------------------
%	HEADERS
%----------------------------------------------------------------------------------------

\renewcommand{\sectionmark}[1]{\markright{\spacedlowsmallcaps{#1}}} % The header for all pages (oneside) or for even pages (twoside)
%\renewcommand{\subsectionmark}[1]{\markright{\thesubsection~#1}} % Uncomment when using the twoside option - this modifies the header on odd pages
\lehead{\mbox{\llap{\small\thepage\kern1em\color{halfgray} \vline}\color{halfgray}\hspace{0.5em}\rightmark\hfil}} % The header style

\pagestyle{scrheadings} % Enable the headers specified in this block

%%%%%%%%%%%%%%%%%%%%%%%%%%%%%%%%%%%%%%%%%
% Academic Title Page
% LaTeX Template
% Version 2.0 (17/7/17)
%
% 
%
%%%%%%%%%%%%%%%%%%%%%%%%%%%%%%%%%%%%%%%%%

%----------------------------------------------------------------------------------------
%	PACKAGES AND OTHER DOCUMENT CONFIGURATIONS
%----------------------------------------------------------------------------------------

%\documentclass[11pt]{article}

%\usepackage[utf8]{inputenc} % Required for inputting international characters
%\usepackage[T1]{fontenc} % Output font encoding for international characters

%\usepackage{mathpazo} % Palatino font

%\begin{document}

%----------------------------------------------------------------------------------------
%	TITLE PAGE
%----------------------------------------------------------------------------------------

\begin{titlepage} % Suppresses displaying the page number on the title page and the subsequent page counts as page 1
	\newcommand{\HRule}{\rule{\linewidth}{0.5mm}} % Defines a new command for horizontal lines, change thickness here
	
	\center % Centre everything on the page
	
	%------------------------------------------------
	%	Headings
	%------------------------------------------------
	
	\textsc{\LARGE WeThinkCode\_}\\[1.5cm] % Main heading such as the name of your university/college
	
	\textsc{\Large WEB II}\\[0.5cm] % Major heading such as course name
	
	\textsc{\large Project II}\\[0.5cm] % Minor heading such as course title
	
	%------------------------------------------------
	%	Title
	%------------------------------------------------
	
	\HRule\\[0.4cm]
	
	{\huge\bfseries Darkly: }\\{\large There is something wrong...}\\[0.4cm] % Title of your document
	
	\HRule\\[1.5cm]
	
	%------------------------------------------------
	%	Author(s)
	%------------------------------------------------
	
	% \begin{minipage}{0.4\textwidth}
	% 	\begin{flushleft}
	% 		\large
	% 		\textit{Developer}\\
	% 		Mosima \textsc{Mamaleka} % Your name
	% 	\end{flushleft}
	% \end{minipage}
	% ~
	% \begin{minipage}{0.4\textwidth}
	% 	\begin{flushright}
	% 		\large
	% 		\textit{Developer}\\
	% 		Sibonelo \textsc{Nkosi} % Supervisor's name
	% 	\end{flushright}
	% \end{minipage}
	
	% If you don't want a supervisor, uncomment the two lines below and comment the code above
	{\large\textit{Developer}}\\
	Sibonelo \textsc{Nkosi}\\ % Your name
	Username: \textsc{sinkosi} % Your name

	\vfill
	{\large\textit{Assessor}}\\
	Mufaro \textsc{Simbisayi}\\ % Your name
	%------------------------------------------------
	%	Date
	%------------------------------------------------
	
	\vfill\vfill\vfill % Position the date 3/4 down the remaining page
	
	\large{October 2020}\\ % Date, change the \today to a set date if you want to be precise
	
	%------------------------------------------------
	%	Logo
	%------------------------------------------------
	
	\vfill\vfill
	
	\includegraphics[width=0.2\textwidth]{00.title/index.png}\\[1cm] % Include a department/university logo - this will require the graphicx package
	\includegraphics[width=0.2\textwidth]{00.title/docker.png}\\[1cm] % Include a department/university logo - this will require the graphicx package
	%----------------------------------------------------------------------------------------
	
	\vfill % Push the date up 1/4 of the remaining page
	
\end{titlepage}

%----------------------------------------------------------------------------------------

%\end{document}


%----------------------------------------------------------------------------------------
%	TABLE OF CONTENTS & LISTS OF FIGURES AND TABLES
%----------------------------------------------------------------------------------------
\newpage
%\maketitle % Print the title/author/date block

\setcounter{tocdepth}{2} % Set the depth of the table of contents to show sections and subsections only

\tableofcontents % Print the table of contents

\newpage

%\listoffigures % Print the list of figures

%\listoftables % Print the list of tables


%----------------------------------------------------------------------------------------
%	ABSTRACT
%----------------------------------------------------------------------------------------

\newpage

\section{Summary}

Right now, you are probably developing a one-block app, with softwares and libraries
installed directly in your development environment, or maybe in a virtual environment...
\textbf{WeThinkCode\_} has provided students with two options.\\

\paragraph{Imagine if your application has to be deployed all over the world and you have to re-
develop it for all existing platforms and OS...}
\subparagraph{Docker was created to satisfy this need for unification and normalisation: it makes it
possible to split an application into several microservices, light, adaptable, universal and
scalable, and it also gives the system administrators a great flexibility to deploy and scale
up the app.}
This suite of projects on Docker will help you better understand this specific tool, but
also the various aspects of applications development using microservices.

The aim of the Docker-1 project is to make you handle docker and docker-machine, the
bases to understand the idea of containerization of services. You can see this project as
an initiation.

\section{Getting Started}

\subsection{Windows}
Windows Installation\textsuperscript{*}: Windows is trash\cite{Docker:Windows_Install}

\subsection{Linux}
\noindent Linux Installation\textsuperscript{**}: Begin by ensuring that you have docker\cite{Docker:Linux_Install} installed on your system, if not type:
\begin{lstlisting}[language=bash]
    $ sudo snap install docker
\end{lstlisting}
%\vfill
\begin{figure}[h]
    \centering
    %\insertcode{Scripts/example.pl}{Nena would be proud.} % The first argument is the script location/filename and the second is a caption for the listing
    %\includegraphics[width=0.752\textwidth]{images/00-0.png}\\[0cm]  
    \caption[Snap Install of Docker]{\emph{sudo snap install},\index{Docker Install} of Docker on Ubuntu}
    %from \url{http://localhost:3000/}).} % The text in the square bracket is the caption for the list of figures while the text in the curly brackets is the figure caption
    \label{fig:00-01 - terminal error VBoxManage} 
\end{figure}

%\vfill
\subsection{MacOS}
At the time of typing this document a Mac was not available to conduct testing
but the documentation\cite{Docker:Mac_Install} does state that installation and setup occurs with a call
to \emph{Homebrew} or an installation of Docker Desktop.
%----------------------------------------------------------------------------------------
%	AUTHOR AFFILIATIONS
%----------------------------------------------------------------------------------------

\let\thefootnote\relax\footnotetext{\textsuperscript{*} \textit{Information provided is correct for current users configuration i.e Windows Home 10:2004, results
may differ for other configurations }}
\let\thefootnote\relax\footnotetext{\textsuperscript{**} \textit{Snap install is not available for all Linux Distros, this is expected to work on Ubuntu and Debian flavours}}

%----------------------------------------------------------------------------------------

%\newpage % Start the article content on the second page, remove this if you have a longer abstract that goes onto the second page


%----------------------------------------------------------------------------------------
%	Flag \#01
%----------------------------------------------------------------------------------------

\section{Flag \#01}
%A statement requiring citation \cite{Figueredo:2009dg}.

%\newpage

%----------------------------------------------------------------------------------------
%	Flag \#02
%----------------------------------------------------------------------------------------

\section{Flag \#02}

%\input{2.method/method.tex}

A Section or subsection covering extensively unit testing will be key either here or on it's own chapter
\newpage
%----------------------------------------------------------------------------------------
%	Flag \#03
%----------------------------------------------------------------------------------------

\section{Flag \#03}

%\input{3.results/result.tex}
\newpage

%----------------------------------------------------------------------------------------
%	Flag \#04
%----------------------------------------------------------------------------------------

\section{Flag \#04}

%Reference to Figure~\vref{fig:snippet}. % The \vref command specifies the location of the reference

%\input{4.code_snip/code.tex}

\newpage

%----------------------------------------------------------------------------------------
%	BIBLIOGRAPHY
%----------------------------------------------------------------------------------------
\section{Bibliography}

\renewcommand{\refname}{\spacedlowsmallcaps{References}} % For modifying the bibliography heading

\bibliographystyle{unsrt}

\bibliography{0.Extras/sample.bib} % The file containing the bibliography
%\bibliography{sample2.bib}

%----------------------------------------------------------------------------------------

%\end{document}


\section{Student Honesty Declaration}

\input{16.cheat/cheat.tex}


%---------------------------------------------------------------------------------------
%\printindex

\end{document}